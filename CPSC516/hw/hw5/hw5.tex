\documentclass[10pt]{article} 

\usepackage{fullpage}
\usepackage{bookmark}

\usepackage[dvipsnames]{xcolor}
\usepackage{amsmath}
\usepackage{amssymb}
\usepackage{mathtools}
\usepackage{mathrsfs}
\usepackage{physics}
% \usepackage{unicode-math}
\usepackage{dsfont}

\usepackage[shortlabels]{enumitem}
\usepackage[noabbrev, nameinlink]{cleveref}
\usepackage[most]{tcolorbox}
\usepackage{empheq}
\usepackage[amsmath,standard,thmmarks]{ntheorem} 
\usepackage{bm}
\usepackage{tabularray}
\usepackage{pdfpages}
\usepackage{float}

\setlist[enumerate]{topsep=1pt,itemsep=0pt,partopsep=1ex,parsep=1ex}

% floor, ceiling, set
\DeclarePairedDelimiter{\ceil}{\lceil}{\rceil}
\DeclarePairedDelimiter{\floor}{\lfloor}{\rfloor}
\DeclarePairedDelimiter{\set}{\lbrace}{\rbrace}
\DeclarePairedDelimiter{\iprod}{\langle}{\rangle}
\DeclarePairedDelimiter{\card}{\lvert}{\rvert}
\let\abs\relax
\DeclarePairedDelimiter{\abs}{\lvert}{\rvert}
\let\norm\relax
\DeclarePairedDelimiter{\norm}{\lVert}{\rVert}

\DeclareMathOperator{\Int}{int}
\DeclareMathOperator{\bdy}{bdy}
\DeclareMathOperator{\Lim}{Lim}
\DeclareMathOperator{\mean}{mean}
\DeclareMathOperator{\col}{col}
\DeclareMathOperator{\proj}{proj}
\DeclareMathOperator{\dual}{dual}
\DeclareMathOperator{\opt}{opt}
\DeclareMathOperator{\cone}{cone}
\DeclareMathOperator{\conv}{conv}
\DeclareMathOperator{\supp}{supp}
\DeclareMathOperator{\poly}{poly}
\DeclareMathOperator{\NB}{NB}
\DeclareMathOperator{\Bin}{Bin}
\DeclareMathOperator{\indeg}{indeg}
\DeclareMathOperator{\outdeg}{outdeg}
\DeclareMathOperator{\Var}{Var}
\DeclareMathOperator{\sgn}{sgn}
\DeclareMathOperator{\Span}{span}
\DeclareMathOperator{\odd}{odd}
\DeclareMathOperator{\OPT}{OPT}
\DeclareMathOperator{\aff}{aff}
\DeclareMathOperator{\ri}{ri}
\DeclareMathOperator{\dom}{dom}
\DeclareMathOperator{\epi}{epi}
\DeclareMathOperator{\Id}{Id}
\DeclareMathOperator{\Fix}{Fix}
\DeclareMathOperator{\Prox}{Prox}
\DeclareMathOperator{\argmin}{argmin}
\DeclareMathOperator{\sign}{sign}
\DeclareMathOperator{\zer}{zer}
\DeclareMathOperator{\fl}{fl}
\DeclareMathOperator{\mach}{mach}
\DeclareMathOperator{\KL}{KL}
\DeclareMathOperator{\diag}{diag}
\DeclareMathOperator{\Softmax}{Softmax}
\DeclareMathOperator{\Po}{Po}
\DeclareMathOperator{\Be}{Be}
\DeclareMathOperator{\Cov}{Cov}
\DeclareMathOperator{\Diag}{Diag}
\DeclareMathOperator{\Op}{op}
\DeclareMathOperator{\adj}{adj}
\DeclareMathOperator{\row}{row}

\newcommand{\code}[1]{\lstinline{#1}}

\newcommand{\ones}{\mathds{1}}

\newcommand{\up}{\uparrow}
\newcommand{\down}{\downarrow}
\newcommand{\tends}[1]{\xrightarrow{#1}}
\newcommand{\eq}[1]{\stackrel{#1}{=}}
\newcommand{\Geq}[1]{\stackrel{#1}{\geq}}
\newcommand{\Leq}[1]{\stackrel{#1}{\leq}}

% commonly used sets
\newcommand{\m}{\mathds{m}}
\newcommand{\R}{\mathbb{R}}
\newcommand{\E}{\mathbb{E}}
\newcommand{\Z}{\mathbb{Z}}
\newcommand{\N}{\mathbb{N}}
\newcommand{\Q}{\mathbb{Q}}
\newcommand{\C}{\mathbb{C}}
\renewcommand{\P}{\mathbb{P}}

\newcommand{\B}{\mathcal{B}}
\newcommand{\F}{\mathcal{F}}
\newcommand{\G}{\mathcal{G}}
\newcommand{\U}{\mathcal{U}}
\newcommand{\I}{\mathcal{I}}
\newcommand{\J}{\mathcal{J}}
\renewcommand{\S}{\mathcal{S}}

\newcommand{\W}{\mathbf{W}}
\newcommand{\w}{\mathbf{w}}
\renewcommand{\c}{\mathbf{c}}
\renewcommand{\d}{\mathbf{d}}
\newcommand{\X}{\mathbf{X}}
\newcommand{\x}{\mathbf{x}}
\newcommand{\Y}{\mathbf{Y}}
\newcommand{\y}{\mathbf{y}}
\newcommand{\z}{\mathbf{z}}
\newcommand{\f}{\mathbf{f}}

\newcommand{\h}{\vec{h}}
\newcommand{\p}{\vec{p}}
\renewcommand{\a}{\vec{a}}
\renewcommand{\b}{\vec{b}}
\renewcommand{\t}{\vec{t}}
\renewcommand{\u}{\vec{u}}
\renewcommand{\v}[1]{\vec{#1}}

\newcommand{\sset}{\subseteq}
\newcommand{\mcal}{\mathcal}
\newcommand{\mscr}{\mathscr}
\newcommand{\mbf}{\mathbf}
\newcommand{\mat}[1]{\begin{bmatrix} #1 \end{bmatrix}}
\newcommand{\eff}{\text{eff}}

\newcommand{\NP}{\ensuremath{\mathcal{NP}}}

\newtcbtheorem[
  auto counter,
  crefname={lemma}{Lemma}
]
{lem}
{Lemma}%
{
  theorem style=break,
  sharp corners=all,
  colframe=Red,
  colback={White!95!Red},
  coltitle=black,
  fonttitle=\bfseries,
  beforeafter skip=12pt
}{lem}

\newtcbtheorem[no counter]{pf}{Proof}%
{
  breakable,
  blanker,
  left=5.5mm,
  borderline west={2pt}{0pt}{NavyBlue!80!white},
  after upper=\null\nobreak\hfill\ensuremath{\square},
  colback=white,
  colframe=white,
  coltitle=black,
  fonttitle=\it,
  parbox=false,
  after skip=12pt
}{pf}

\definecolor{dkgreen}{rgb}{0,0.6,0}
\definecolor{gray}{rgb}{0.5,0.5,0.5}
\definecolor{mauve}{rgb}{0.58,0,0.82}
\lstset{
    frame=tb,
    language=Python,
    aboveskip=3mm,
    belowskip=3mm,
    showstringspaces=false,
    columns=flexible,
    basicstyle={\small\ttfamily},
    numbers=none,
    numberstyle=\tiny\color{gray},
    keywordstyle=\color{blue},
    commentstyle=\color{dkgreen},
    stringstyle=\color{mauve},
    breaklines=true,
    breakatwhitespace=true,
    tabsize=3
}

\setlength\parindent{0pt}
\setlength{\parskip}{8pt}

\setcounter{secnumdepth}{2}
\renewcommand\thesection{P.\arabic{section}.}
\renewcommand\thesubsection{(\alph{subsection})}


\begin{document}

\begin{center}
    {\Large\textbf{Yale University}}\\
    \vspace{3mm}
    {\Large\textbf{CPSC 516, Spring 2023}}\\
    \vspace{2mm}
    {\Large\textbf{Assignment 5}}\\
    \vspace{3mm}
    \textbf{Chang Feng (Felix) Zhou cz397}
\end{center}

\section{}
\subsection{}
From our work in class,
we know that $\grad^2 f\succeq mI$ implies that $f$ is $m$-strongly convex.
This yields the inequality
\[
  f(y) - [f(x) + \iprod{\grad f(x), y-x}] \geq \frac{m}2 \norm{y-x}_2^2
\]
by the definition of strong convexity.

On the other hand,
we know by the second order Taylor expansion about $x$ that
\begin{align*}
  f(y)
  &= f(x) + \iprod{\grad f(x), y-x} + \frac12 (y-x)^T \grad^2 f(\xi) (y-x) &&\xi\in [x, y] \\
  f(y) - f(x) + \iprod{\grad f(x), x-y}
  &\leq \frac{M}2 \norm{y-x}_2^2. &&MI \succeq \grad^2 f
\end{align*}
This shows both inequalities.

\subsection{}
Suppose $f(z^*) = y^*$
and consider the inequality from P.1.(a)
\[
  y^*\leq f(z) \leq f(x) + \iprod{\grad f(x), z-x} + \frac{M}2 \norm{z-x}_2^2.
\]
Let us minimize the RHS with respect to $z$ by taking the derivative and setting it to 0.
We must have
\begin{align*}
  \grad f(x) + M[z-x] &= 0 \\
  z &= x - \frac1M \grad f(x).
\end{align*}
Substituting this particular value of $z$ to the RHS above yields
\begin{align*}
  &f(x) + \iprod*{\grad f(x), -\frac1M \grad f(x)} + \frac1{2M} \norm{\grad f(x)}_2^2 \\
  &= f(x) - \frac1{2M} \norm{\grad f(x)}_2^2
\end{align*}
which is one of the desired inequalities.

To see the other inequality,
We note that we wish to prove
\[
  \frac1{2m} \norm{\grad f(x)}_2^2
  \geq [f(x) - f(z^*)]
\]
which is known as the \emph{Polyak-Lojasiewicz (PL)} condition.

By the definition of strong convexity,
\begin{align*}
  f(x) - f(z^*)
  &\leq \iprod{\grad f(x), x-z^*} - \frac{m}2 \norm{x-z^*}_2^2 \\
  &= \iprod{\grad f(x), x-z^*} - \frac{m}2 \norm{x-z^*}_2^2 - \frac1{2m} \norm{\grad f(x)}_2^2 + \frac1{2m} \norm{\grad f(x)}_2^2 \\
  &= -\frac12 \norm*{\sqrt m (x-z^*) - \frac1{\sqrt m} \grad f(x)}_2^2 + \frac1{2m} \norm{\grad f(x)}_2^2 \\
  &\leq \frac1{2m} \norm{\grad f(x)}_2^2.
\end{align*}

Having shown both inequalities,
we conclude the proof.

\subsection{}
In P.1.(b),
we have shown that
\[
  f(z) \leq f(x) - \frac1{2M} \norm{\grad f(x)}_2^2
\]
for $z := x - \frac1M \grad f(x)$.

But since we chose the step size $\alpha$ to minimize $f(x_{t+1})$,
it must be at least as good as $\alpha = \frac1M$.
Thus
\[
  f(x_{t+1})\leq f(x_t) - \frac1M \norm{\grad f(x_t)}_2^2
\]
as desired.

\subsection{}
We argue by induction,
the base case
\[
  f(x_0) - y^* \leq 1\cdot [f(x_0) - y^*]
\]
holds trivially.
Suppose it holds up to some $t$
and consider $f(x_{t+1}) - y^*$.

We have
\begin{align*}
  f(x_{t+1}) - y^*
  &= f(x_{t+1}) - f(x_t) + f(x_t) - y^* \\
  &\leq f(x_t) - y^* - \frac1{2M} \norm{\grad f(x_t)}_2^2 &&\text{P.1.(c)} \\
  &\leq f(x_t) - y^* - \frac{m}M [f(x_t) - y^*] &&\text{P.1.(b) LHS} \\
  &= \left( 1-\frac{m}M \right) [f(x_t) - y^*] \\
  &= \left( 1-\frac{m}M \right)^{t+1} [f(x_0) - y^*]. &&\text{induction hypothesis}
\end{align*}

By induction,
we conclude the proof.

The number of iterations to reach $\varepsilon$ error can be computed as follows
\begin{align*}
  \left( 1-\frac{m}M \right)^t [f(x_0) - y^*]
  &\leq \exp(-mt/M) [f(x_0) - y^*] &&1-x\leq e^{-x} \\
  &\leq \varepsilon \\
  -\frac{mt}{M} + \log[f(x_0) - y^*] &\leq \log \varepsilon \\
  t &\geq \frac{M}{m} \log\frac{f(x_0) - y^*}{\varepsilon}.
\end{align*}

\subsection{}
Suppose we are given $A, b$ as input.

Consider the minimization problem
\begin{align*}
  &\min \norm{Ax - b}_2^2 \\
  x &\in \R^n
\end{align*}

The objective is the composition of an affine function
and a convex, separable, and non-decreasing (in each coordinate) function,
which is therefore convex.

We explicitly compute its first and second derivatives
\begin{align*}
  \frac{d}{dx} [Ax+b]^T[Ax + b]
  &= \frac{d}{dx} xA^2x + 2x^TAb + b^Tb \\
  &= 2 A^2x + 2Ab \\
  \frac{d^2}{dx^2}
  &= 2A^2.
\end{align*}
The objective is certainly twice differentiable,
and since the eigenvalues of $A^2$ are just the eigenvalues of $A$ squared,
\[
  \lambda_1(A)^2 \leq \grad f^2 \leq \lambda_n(A)^2.
\]

By our work above,
if we start with an initial solution $x_0 := 0$
and run gradient descent with step size $\alpha = \frac1{\lambda_n(A)^2}$,
this yields a solution $x$ such that $\norm{Ax-b}_2^2 \leq \varepsilon$ after
\[
  T = O\left( \frac{\lambda_n(A)^2}{\lambda_1(A)^2} \log\frac{\norm{b}_2^2}{\varepsilon} \right)
\]
iterations.
In each iteration,
we need to compute the gradient and subtract it from the current iterate.
The number of arithmetic operations is dominated by the gradient computation $A(Ax)$,
which requires $O(n^2)$ operations if we compute $Ax$ and then $A(Ax)$.

Thus the algorithm terminates after performing
\[
  O(n^2 T)
  = O\left( n^2 \kappa^2 \log \frac{\norm{b}_2^2}{\varepsilon} \right)
\]
arithmetic operations.

\clearpage
\section{}
\begin{lem}{}{least-squares}
  $(BB^T)^\dagger Bg$ is a minimizer of the optimization problem
  \[
    \min_{y\in \R^n} \norm{B^Ty - g}_2^2.
  \]
  Thus $B^T(BB^T)^\dagger Bg$ is the Euclidean projection of $g$ onto the row space of $B$.
\end{lem}

\begin{pf}{\Cref{lem:least-squares}}{}
  The objective function is a composition of an affine (convex) function
  with a convex, separable, and coordinate-wise non-decreasing function,
  which is therefore convex.
  We can thus solve this problem by taking the gradient and setting it to zero.

  The objective is equivalent to
  \begin{align*}
    (B^Ty - g)^T(B^Ty - g)
    &= y^TBB^Ty - 2g^TB^Ty - g^Tg.
  \end{align*}
  Taking the derivative and setting it to 0 yields
  \begin{align*}
    2y^TBB^T - 2g^TB^T &= 0 \\
    y &= (BB^T)^\dagger Bg.
  \end{align*}
\end{pf}

Recall from elmentary linear algebra that the kernel is the orthogonal complement of the rowspace
so that we can write
\[
  \R^m = \ker B \oplus \row(B).
\]
In particular,
if $r := \rank B$,
we can find an orthonormal basis of $\R^m$,
say $v_1, \dots, v_r, w_1, \dots, w_{m-r}$,
where $v_i\in \row(B), w_j\in \ker B$.
Thus we can write
\[
  g = \sum_{i=1}^r \iprod{g, v_i} v_i + \sum_{j=1}^{m-r} \iprod{g, w_j} w_j.
\]

By elementary linear algebra,
\begin{align*}
  x_g &= \sum_{i=1}^r \iprod{g, v_i} v_i \\
  \Pi g &= \sum_{j=1}^{m-r} \iprod{g, w_j} w_j.
\end{align*}
Thus
\[
  x_g + \Pi g = g
\]
as desired.

\end{document}
