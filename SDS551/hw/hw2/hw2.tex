\documentclass[10pt]{article} 

\usepackage{fullpage}
\usepackage{bookmark}

\usepackage[dvipsnames]{xcolor}
\usepackage{amsmath}
\usepackage{amssymb}
\usepackage{mathtools}
\usepackage{mathrsfs}
\usepackage{physics}
% \usepackage{unicode-math}
\usepackage{dsfont}

\usepackage[shortlabels]{enumitem}
\usepackage[noabbrev, nameinlink]{cleveref}
\usepackage[most]{tcolorbox}
\usepackage{empheq}
\usepackage[amsmath,standard,thmmarks]{ntheorem} 
\usepackage{bm}
\usepackage{tabularray}
\usepackage{pdfpages}
\usepackage{float}

\setlist[enumerate]{topsep=1pt,itemsep=0pt,partopsep=1ex,parsep=1ex}

% floor, ceiling, set
\DeclarePairedDelimiter{\ceil}{\lceil}{\rceil}
\DeclarePairedDelimiter{\floor}{\lfloor}{\rfloor}
\DeclarePairedDelimiter{\set}{\lbrace}{\rbrace}
\DeclarePairedDelimiter{\iprod}{\langle}{\rangle}
\DeclarePairedDelimiter{\card}{\lvert}{\rvert}
\let\abs\relax
\DeclarePairedDelimiter{\abs}{\lvert}{\rvert}
\let\norm\relax
\DeclarePairedDelimiter{\norm}{\lVert}{\rVert}

\DeclareMathOperator{\Int}{int}
\DeclareMathOperator{\bdy}{bdy}
\DeclareMathOperator{\Lim}{Lim}
\DeclareMathOperator{\mean}{mean}
\DeclareMathOperator{\col}{col}
\DeclareMathOperator{\proj}{proj}
\DeclareMathOperator{\dual}{dual}
\DeclareMathOperator{\opt}{opt}
\DeclareMathOperator{\cone}{cone}
\DeclareMathOperator{\conv}{conv}
\DeclareMathOperator{\supp}{supp}
\DeclareMathOperator{\poly}{poly}
\DeclareMathOperator{\NB}{NB}
\DeclareMathOperator{\Bin}{Bin}
\DeclareMathOperator{\indeg}{indeg}
\DeclareMathOperator{\outdeg}{outdeg}
\DeclareMathOperator{\Var}{Var}
\DeclareMathOperator{\sgn}{sgn}
\DeclareMathOperator{\Span}{span}
\DeclareMathOperator{\odd}{odd}
\DeclareMathOperator{\OPT}{OPT}
\DeclareMathOperator{\aff}{aff}
\DeclareMathOperator{\ri}{ri}
\DeclareMathOperator{\dom}{dom}
\DeclareMathOperator{\epi}{epi}
\DeclareMathOperator{\Id}{Id}
\DeclareMathOperator{\Fix}{Fix}
\DeclareMathOperator{\Prox}{Prox}
\DeclareMathOperator{\argmin}{argmin}
\DeclareMathOperator{\sign}{sign}
\DeclareMathOperator{\zer}{zer}
\DeclareMathOperator{\fl}{fl}
\DeclareMathOperator{\mach}{mach}
\DeclareMathOperator{\KL}{KL}
\DeclareMathOperator{\diag}{diag}
\DeclareMathOperator{\Softmax}{Softmax}
\DeclareMathOperator{\Po}{Po}
\DeclareMathOperator{\Be}{Be}
\DeclareMathOperator{\Cov}{Cov}

\newcommand{\code}[1]{\lstinline{#1}}

\newcommand{\ones}{\mathds{1}}

\newcommand{\up}{\uparrow}
\newcommand{\down}{\downarrow}
\newcommand{\tends}[1]{\xrightarrow{#1}}
\newcommand{\eq}[1]{\stackrel{#1}{=}}
\newcommand{\Geq}[1]{\stackrel{#1}{\geq}}
\newcommand{\Leq}[1]{\stackrel{#1}{\leq}}

% commonly used sets
\newcommand{\m}{\mathds{m}}
\newcommand{\R}{\mathbb{R}}
\newcommand{\E}{\mathbb{E}}
\newcommand{\Z}{\mathbb{Z}}
\newcommand{\N}{\mathbb{N}}
\newcommand{\Q}{\mathbb{Q}}
\newcommand{\C}{\mathbb{C}}
\renewcommand{\P}{\mathbb{P}}

\newcommand{\B}{\mathcal{B}}
\newcommand{\F}{\mathcal{F}}
\newcommand{\G}{\mathcal{G}}
\newcommand{\U}{\mathcal{U}}
\newcommand{\I}{\mathcal{I}}
\newcommand{\J}{\mathcal{J}}
\renewcommand{\S}{\mathcal{S}}

\newcommand{\W}{\mathbf{W}}
\newcommand{\w}{\mathbf{w}}
\renewcommand{\c}{\mathbf{c}}
\renewcommand{\d}{\mathbf{d}}
\newcommand{\X}{\mathbf{X}}
\newcommand{\x}{\mathbf{x}}
\newcommand{\Y}{\mathbf{Y}}
\newcommand{\y}{\mathbf{y}}
\newcommand{\z}{\mathbf{z}}
\newcommand{\f}{\mathbf{f}}

\newcommand{\h}{\vec{h}}
\newcommand{\p}{\vec{p}}
\renewcommand{\a}{\vec{a}}
\renewcommand{\b}{\vec{b}}
\renewcommand{\t}{\vec{t}}
\renewcommand{\u}{\vec{u}}
\renewcommand{\v}[1]{\vec{#1}}

\newcommand{\sset}{\subseteq}
\newcommand{\mcal}{\mathcal}
\newcommand{\mscr}{\mathscr}
\newcommand{\mbf}{\mathbf}
\newcommand{\mat}[1]{\begin{bmatrix} #1 \end{bmatrix}}
\newcommand{\eff}{\text{eff}}

\newcommand{\NP}{\ensuremath{\mathcal{NP}}}

\newtcbtheorem[
  number within=section,
  crefname={lemma}{Lemma}
]
{lem}
{Lemma}%
{
  theorem style=break,
  sharp corners=all,
  colframe=Red,
  colback={White!95!Red},
  coltitle=black,
  fonttitle=\bfseries,
  beforeafter skip=12pt
}{lem}

\newtcbtheorem[no counter]{pf}{Proof}%
{
  breakable,
  blanker,
  left=5.5mm,
  borderline west={2pt}{0pt}{NavyBlue!80!white},
  after upper=\null\nobreak\hfill\ensuremath{\square},
  colback=white,
  colframe=white,
  coltitle=black,
  fonttitle=\it,
  parbox=false,
  after skip=12pt
}{pf}

\definecolor{dkgreen}{rgb}{0,0.6,0}
\definecolor{gray}{rgb}{0.5,0.5,0.5}
\definecolor{mauve}{rgb}{0.58,0,0.82}
\lstset{
    frame=tb,
    language=Python,
    aboveskip=3mm,
    belowskip=3mm,
    showstringspaces=false,
    columns=flexible,
    basicstyle={\small\ttfamily},
    numbers=none,
    numberstyle=\tiny\color{gray},
    keywordstyle=\color{blue},
    commentstyle=\color{dkgreen},
    stringstyle=\color{mauve},
    breaklines=true,
    breakatwhitespace=true,
    tabsize=3
}

\setlength\parindent{0pt}
\setlength{\parskip}{8pt}

\setcounter{secnumdepth}{2}
\renewcommand\thesection{Problem \arabic{section}.}
\renewcommand\thesubsection{(\arabic{subsection})}


\begin{document}

\begin{center}
    {\Large\textbf{Yale University}}\\
    \vspace{3mm}
    {\Large\textbf{S\&DS 551, Spring 2023}}\\
    \vspace{2mm}
    {\Large\textbf{Homework 2}}\\
    \vspace{3mm}
    \textbf{Chang Feng (Felix) Zhou cz397}
\end{center}

\section{}
First,
we note that the random walk $S$ can only return to 0
at an even time step.

We wish to count the number of binary strings of length $2n$
that is \emph{balanced} (equal number of 0s and 1s)
subject to the condition that any prefix is not balanced.
This corresponds to a random walk which returns to 0 at time $2n$
but not before that.

By symmetry,
it suffices to double the number of binary strings
on $n$ 1s and $n-1$ 0s such that any prefix contains strictly more 1s than 0s
(the $0$-th character is prepended as 0 to ensure balance).
We claim that there are
\[
  \frac1{2n-1} \binom{2n-1}{n-1}
\]
such strings.

Before we prove the claim.
Note that if the claim holds,
then we are done.
This is because the desired probability is then given by
\begin{align*}
  \frac{\frac1{2n-1} \binom{2n-1}{n-1} \cdot 2}{2^{2n}}
  &= \frac1{2n-1} \binom{2n}{n} 2^{-2n}
\end{align*}
as required.

To see the claim,
we observe that there is a bijection between 
the number of binary strings
on $n$ 1s and $n-1$ 0s such that any prefix contains strictly more 1s than 0s
and the number of binary strings
on $n$ 1s and $n$ 0s such that any prefix contains at least as many 1s as 0s.
Indeed,
the bijection is obtained by prepending a 0 to the $(2n-1)$-bit string.
But the cardinality of the latter set
is precisely given by the well-known $(n-1)$-th Catalan number
\[
  C_n
  = \frac1{2n-1} \binom{2n-1}{n-1}
  = \frac1{2n-1} \binom{2n-1}{n}.
\]

Finally,
we have
\begin{align*}
  \E[T^\alpha]
  &= \sum_{n=1}^\infty \P\set{T=2n}\cdot (2n)^\alpha \\
  &= \sum_{n=1}^\infty \frac{(2n)^\alpha}{2n-1} \binom{2n}{n} 2^{-2n} \\
  &= \sum_{n=1}^\infty \frac{(2n)^\alpha}{2n-1} \frac{(2n)!}{n!n!} 2^{-2n} \\
  &\approx \sum_{n=1}^\infty \frac1{2n-1} \frac{(2n)^{2n+\frac12+\alpha} e^{-2n} \sqrt{2\pi}}{\left( n^{n+\frac12} e^{-n} \sqrt{2\pi} \right)^2} 2^{-2n} \\
  &= \frac1{\sqrt{2\pi}} \sum_{n=1}^\infty \frac1{2n-1} \frac{(2n)^{2n+\frac12+\alpha}}{n^{2n+1}} 2^{-2n} \\
  &= \frac{2^{\frac12+\alpha}}{\sqrt{2\pi}} \sum_{n=1}^\infty \frac1{2n-1} n^{\alpha-\frac12}.
\end{align*}

For $\alpha < \frac12$,
this is series is bounded above by a convergent $p$-series
\[
  \sum_{n\geq 1} \frac1{n^p}
\]
for some $p > 1$.
On the other hand,
for $\alpha \geq \frac12$,
this series is bounded below by the divergent harmonic series
\[
  \sum_{n\geq 1} \frac1n.
\]

This concludes the proof.

\clearpage
\section{}
We have that
\begin{align*}
  \P\set{R_{n} = R_{n-1} + 1}
  &= \P\set{S_n\neq S_{n-1}, \dots, S_n\neq S_0} \\
  &= \P\set{S_1\neq 0, \dots, S_n\neq 0} \\
  &= \P\set{S_1S_2\dots S_n\neq 0}.
\end{align*}
Note that here we used the fact
\[
  \sum_{i=1}^k X_i \eq{d} \sum_{i=1}^k X_{n-i+1}
\]
since the $X_i$'s are iid.

Now let $x_n := \E[R_n]$ and $p_n := \P\set{R_n = R_{n-1} + 1}$.
By the downward continuity of measure,
we have that as $n\to \infty$,
\[
  p_n\to \ell =: \P\set{\forall k\geq 1, S_k=0}.
\]

By the law of total expectation,
\begin{align*}
  x_n
  &= \E[R_n\mid R_n=R_{n-1}]\cdot (1-p_n) + \E[R_n\mid R_n=R_{n-1}+1]\cdot p_n \\
  &= x_{n-1} (1-p) + (x_{n-1} + 1) p_n \\
  &= x_{n-1} + p_n \\
  &= \sum_{k=1}^n p_k.
\end{align*}

Recall that if a real-valued sequence converges,
then the average of the partial sums also converge to the same limit.
It follows that
\[
  \frac1n \E[R_n]
  = \frac1n \sum_{k=1}^n p_k
  \to \ell
\]
as $n\to \infty$.

Finally,
we wish to show that in the case of simple random walks,
\[
  \ell = \abs{p-q}.
\]
Without loss of generality,
let us assume that $p\geq q$
and show that $\ell = p-q$.

Equivalently,
we can show that the probability of eventually hitting 0 is $2q$.
Then $\ell = 1-2q = p-q$.
Let $p_k$ denote the probability of eventually hitting 0
starting at position $k > 0$.
Then we have the recurrence
\begin{align*}
  p_1
  &= q + p p_2 \\
  &= q + p p_1^2.
\end{align*}
To see this remark that the increments of the random walk are independent,
thus to arrive at 0 starting at 2
is equivalent to two independent random walks hitting 0 when starting at 1.

Solving this quadratic equation yields solutions $1, \frac qp$.
In the case of $p\geq q$,
we take $\frac qp$
and in the case of $p < q$,
we take 1.
The probability of eventually returning to 0
assuming that $p\geq q$ is thus
\[
  q\cdot 1 + p\cdot \frac qp = 2q
\]
as desired.
To see this,
we condition on the first step being to the left or the right
and apply the appropriate solution to the quadratic equation above.

\clearpage
\section{}
From our work in class,
we know that the expected hitting time for $p=q=\frac12$ is infinity.
But then the expected hitting time for $p<q$ can only be greater
and is thus not finite either.
We focus on the case where $p>q$.

Let us rewrite this as the expected hitting time of 0 starting at position $b$
by flipping the values of $p, q$.
Our strategy to compute this value is to compute the expected hitting time by setting an artificial boundary at $n$
and then letting $n\to \infty$,
similar to our work in class.
This strategy is justified by the monontone convergence theorem.

Let $T_{b, n}$ denote the hitting time of one of two barriers $0, n$
starting at $b\in [0, n]$.
We have
\begin{align*}
  \E[T_{b, n}]
  &= q\E[T_{b-1, n}] + p\E[T_{b+1, n}] + 1 \\
  \E[T_{0, n}] &= 0 \\
  \E[T_{n, n}] &= 0.
\end{align*}

Solving this recurrence under the boundary conditions yield
\begin{align*}
  \E[T_{b, n}]
  &= \frac{b}{q-p} - \frac{n}{q-p}\cdot \frac{\left( \frac qp \right)^b - 1}{\left( \frac qp \right)^n - 1} \\
  &\to \frac{b}{q-p}. &&n\to \infty
\end{align*}

Keeping in mind that we flipped the values of $p, q$,
we conclude that the desired expectation is
\[
  \frac{b}{p-q}
\]
as desired.

\clearpage
\section{}
\section{}

\end{document}
