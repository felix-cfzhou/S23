\documentclass[10pt]{article} 

\usepackage{fullpage}
\usepackage{bookmark}

\usepackage[dvipsnames]{xcolor}
\usepackage{amsmath}
\usepackage{amssymb}
\usepackage{mathtools}
\usepackage{mathrsfs}
\usepackage{physics}
% \usepackage{unicode-math}
\usepackage{dsfont}

\usepackage[shortlabels]{enumitem}
\usepackage[noabbrev, nameinlink]{cleveref}
\usepackage[most]{tcolorbox}
\usepackage{empheq}
\usepackage[amsmath,standard,thmmarks]{ntheorem} 
\usepackage{bm}
\usepackage{tabularray}
\usepackage{pdfpages}
\usepackage{float}

\setlist[enumerate]{topsep=1pt,itemsep=0pt,partopsep=1ex,parsep=1ex}

% floor, ceiling, set
\DeclarePairedDelimiter{\ceil}{\lceil}{\rceil}
\DeclarePairedDelimiter{\floor}{\lfloor}{\rfloor}
\DeclarePairedDelimiter{\set}{\lbrace}{\rbrace}
\DeclarePairedDelimiter{\iprod}{\langle}{\rangle}
\DeclarePairedDelimiter{\card}{\lvert}{\rvert}
\let\abs\relax
\DeclarePairedDelimiter{\abs}{\lvert}{\rvert}
\let\norm\relax
\DeclarePairedDelimiter{\norm}{\lVert}{\rVert}

\DeclareMathOperator{\Int}{int}
\DeclareMathOperator{\bdy}{bdy}
\DeclareMathOperator{\Lim}{Lim}
\DeclareMathOperator{\mean}{mean}
\DeclareMathOperator{\col}{col}
\DeclareMathOperator{\proj}{proj}
\DeclareMathOperator{\dual}{dual}
\DeclareMathOperator{\opt}{opt}
\DeclareMathOperator{\cone}{cone}
\DeclareMathOperator{\conv}{conv}
\DeclareMathOperator{\supp}{supp}
\DeclareMathOperator{\poly}{poly}
\DeclareMathOperator{\NB}{NB}
\DeclareMathOperator{\Bin}{Bin}
\DeclareMathOperator{\indeg}{indeg}
\DeclareMathOperator{\outdeg}{outdeg}
\DeclareMathOperator{\Var}{Var}
\DeclareMathOperator{\sgn}{sgn}
\DeclareMathOperator{\Span}{span}
\DeclareMathOperator{\odd}{odd}
\DeclareMathOperator{\OPT}{OPT}
\DeclareMathOperator{\aff}{aff}
\DeclareMathOperator{\ri}{ri}
\DeclareMathOperator{\dom}{dom}
\DeclareMathOperator{\epi}{epi}
\DeclareMathOperator{\Id}{Id}
\DeclareMathOperator{\Fix}{Fix}
\DeclareMathOperator{\Prox}{Prox}
\DeclareMathOperator{\argmin}{argmin}
\DeclareMathOperator{\sign}{sign}
\DeclareMathOperator{\zer}{zer}
\DeclareMathOperator{\fl}{fl}
\DeclareMathOperator{\mach}{mach}
\DeclareMathOperator{\KL}{KL}
\DeclareMathOperator{\diag}{diag}
\DeclareMathOperator{\Softmax}{Softmax}
\DeclareMathOperator{\Po}{Po}
\DeclareMathOperator{\Be}{Be}
\DeclareMathOperator{\Cov}{Cov}

\newcommand{\code}[1]{\lstinline{#1}}

\newcommand{\ones}{\mathds{1}}

\newcommand{\up}{\uparrow}
\newcommand{\down}{\downarrow}
\newcommand{\tends}[1]{\xrightarrow{#1}}
\newcommand{\eq}[1]{\stackrel{#1}{=}}
\newcommand{\Geq}[1]{\stackrel{#1}{\geq}}
\newcommand{\Leq}[1]{\stackrel{#1}{\leq}}

% commonly used sets
\newcommand{\m}{\mathds{m}}
\newcommand{\R}{\mathbb{R}}
\newcommand{\E}{\mathbb{E}}
\newcommand{\Z}{\mathbb{Z}}
\newcommand{\N}{\mathbb{N}}
\newcommand{\Q}{\mathbb{Q}}
\newcommand{\C}{\mathbb{C}}
\renewcommand{\P}{\mathbb{P}}

\newcommand{\B}{\mathcal{B}}
\newcommand{\F}{\mathcal{F}}
\newcommand{\G}{\mathcal{G}}
\newcommand{\U}{\mathcal{U}}
\newcommand{\I}{\mathcal{I}}
\newcommand{\J}{\mathcal{J}}
\renewcommand{\S}{\mathcal{S}}

\newcommand{\W}{\mathbf{W}}
\newcommand{\w}{\mathbf{w}}
\renewcommand{\c}{\mathbf{c}}
\renewcommand{\d}{\mathbf{d}}
\newcommand{\X}{\mathbf{X}}
\newcommand{\x}{\mathbf{x}}
\newcommand{\Y}{\mathbf{Y}}
\newcommand{\y}{\mathbf{y}}
\newcommand{\z}{\mathbf{z}}
\newcommand{\f}{\mathbf{f}}

\newcommand{\h}{\vec{h}}
\newcommand{\p}{\vec{p}}
\renewcommand{\a}{\vec{a}}
\renewcommand{\b}{\vec{b}}
\renewcommand{\t}{\vec{t}}
\renewcommand{\u}{\vec{u}}
\renewcommand{\v}[1]{\vec{#1}}

\newcommand{\sset}{\subseteq}
\newcommand{\mcal}{\mathcal}
\newcommand{\mscr}{\mathscr}
\newcommand{\mbf}{\mathbf}
\newcommand{\mat}[1]{\begin{bmatrix} #1 \end{bmatrix}}
\newcommand{\eff}{\text{eff}}

\newcommand{\NP}{\ensuremath{\mathcal{NP}}}

\newtcbtheorem[
  number within=section,
  crefname={lemma}{Lemma}
]
{lem}
{Lemma}%
{
  theorem style=break,
  sharp corners=all,
  colframe=Red,
  colback={White!95!Red},
  coltitle=black,
  fonttitle=\bfseries,
  beforeafter skip=12pt
}{lem}

\newtcbtheorem[no counter]{pf}{Proof}%
{
  breakable,
  blanker,
  left=5.5mm,
  borderline west={2pt}{0pt}{NavyBlue!80!white},
  after upper=\null\nobreak\hfill\ensuremath{\square},
  colback=white,
  colframe=white,
  coltitle=black,
  fonttitle=\it,
  parbox=false,
  after skip=12pt
}{pf}

\definecolor{dkgreen}{rgb}{0,0.6,0}
\definecolor{gray}{rgb}{0.5,0.5,0.5}
\definecolor{mauve}{rgb}{0.58,0,0.82}
\lstset{
    frame=tb,
    language=Python,
    aboveskip=3mm,
    belowskip=3mm,
    showstringspaces=false,
    columns=flexible,
    basicstyle={\small\ttfamily},
    numbers=none,
    numberstyle=\tiny\color{gray},
    keywordstyle=\color{blue},
    commentstyle=\color{dkgreen},
    stringstyle=\color{mauve},
    breaklines=true,
    breakatwhitespace=true,
    tabsize=3
}

\setlength\parindent{0pt}
\setlength{\parskip}{8pt}

\setcounter{secnumdepth}{2}
\renewcommand\thesection{Problem \arabic{section}.}
\renewcommand\thesubsection{(\arabic{subsection})}


\begin{document}

\begin{center}
    {\Large\textbf{Yale University}}\\
    \vspace{3mm}
    {\Large\textbf{S\&DS 551, Spring 2023}}\\
    \vspace{2mm}
    {\Large\textbf{Homework 2}}\\
    \vspace{3mm}
    \textbf{Chang Feng (Felix) Zhou cz397}
\end{center}

\section{}
First,
we note that the random walk $S$ can only return to 0
at an even time step.

We wish to count the number of binary strings of length $2n$
that is \emph{balanced} (equal number of 0s and 1s)
subject to the condition that any prefix is not balanced.
This corresponds to a random walk which returns to 0 at time $2n$
but not before that.

By symmetry,
it suffices to double the number of binary strings
on $n$ 1s and $n-1$ 0s such that any prefix contains strictly more 1s than 0s
(the $0$-th character is prepended as 0 to ensure balance).
We claim that there are
\[
  \frac1{2n-1} \binom{2n-1}{n-1}
\]
such strings.

Before we prove the claim.
Note that if the claim holds,
then we are done.
This is because the desired probability is then given by
\begin{align*}
  \frac{\frac1{2n-1} \binom{2n-1}{n-1} \cdot 2}{2^{2n}}
  &= \frac1{2n-1} \binom{2n}{n} 2^{-2n}
\end{align*}
as required.

To see the claim,
we observe that there is a bijection between 
the number of binary strings
on $n$ 1s and $n-1$ 0s such that any prefix contains strictly more 1s than 0s
and the number of binary strings
on $n$ 1s and $n$ 0s such that any prefix contains at least as many 1s as 0s.
Indeed,
the bijection is obtained by prepending a 0 to the $(2n-1)$-bit string.
But the cardinality of the latter set
is precisely given by the well-known $(n-1)$-th Catalan number
\[
  C_n
  = \frac1{2n-1} \binom{2n-1}{n-1}
  = \frac1{2n-1} \binom{2n-1}{n}.
\]

Finally,
we have
\begin{align*}
  \E[T^\alpha]
  &= \sum_{n=1}^\infty \P\set{T=2n}\cdot (2n)^\alpha \\
  &= \sum_{n=1}^\infty \frac{(2n)^\alpha}{2n-1} \binom{2n}{n} 2^{-2n} \\
  &= \sum_{n=1}^\infty \frac{(2n)^\alpha}{2n-1} \frac{(2n)!}{n!n!} 2^{-2n} \\
  &\approx \sum_{n=1}^\infty \frac1{2n-1} \frac{(2n)^{2n+\frac12+\alpha} e^{-2n} \sqrt{2\pi}}{\left( n^{n+\frac12} e^{-n} \sqrt{2\pi} \right)^2} 2^{-2n} \\
  &= \frac1{\sqrt{2\pi}} \sum_{n=1}^\infty \frac1{2n-1} \frac{(2n)^{2n+\frac12+\alpha}}{n^{2n+1}} 2^{-2n} \\
  &= \frac{2^{\frac12+\alpha}}{\sqrt{2\pi}} \sum_{n=1}^\infty \frac1{2n-1} n^{\alpha-\frac12}.
\end{align*}

For $\alpha < \frac12$,
this is series is bounded above by a convergent $p$-series
\[
  \sum_{n\geq 1} \frac1{n^p}
\]
for some $p > 1$.
On the other hand,
for $\alpha \geq \frac12$,
this series is bounded below by the divergent harmonic series
\[
  \sum_{n\geq 1} \frac1n.
\]

This concludes the proof.

\clearpage
\section{}
We have that
\begin{align*}
  \P\set{R_{n} = R_{n-1} + 1}
  &= \P\set{S_n\neq S_{n-1}, \dots, S_n\neq S_0} \\
  &= \P\set{S_1\neq 0, \dots, S_n\neq 0} \\
  &= \P\set{S_1S_2\dots S_n\neq 0}.
\end{align*}
Note that here we used the fact
\[
  \sum_{i=1}^k X_i \eq{d} \sum_{i=1}^k X_{n-i+1}
\]
since the $X_i$'s are iid.

Now let $x_n := \E[R_n]$ and $p_n := \P\set{R_n = R_{n-1} + 1}$.
By the downward continuity of measure,
we have that as $n\to \infty$,
\[
  p_n\to \ell =: \P\set{\forall k\geq 1, S_k=0}.
\]

By the law of total expectation,
\begin{align*}
  x_n
  &= \E[R_n\mid R_n=R_{n-1}]\cdot (1-p_n) + \E[R_n\mid R_n=R_{n-1}+1]\cdot p_n \\
  &= x_{n-1} (1-p) + (x_{n-1} + 1) p_n \\
  &= x_{n-1} + p_n \\
  &= \sum_{k=1}^n p_k.
\end{align*}

Recall that if a real-valued sequence converges,
then the average of the partial sums also converge to the same limit.
It follows that
\[
  \frac1n \E[R_n]
  = \frac1n \sum_{k=1}^n p_k
  \to \ell
\]
as $n\to \infty$.

Finally,
we wish to show that in the case of simple random walks,
\[
  \ell = \abs{p-q}.
\]
Without loss of generality,
let us assume that $p\geq q$
and show that $\ell = p-q$.

Equivalently,
we can show that the probability of eventually hitting 0 is $2q$.
Then $\ell = 1-2q = p-q$.
Let $p_k$ denote the probability of eventually hitting 0
starting at position $k > 0$.
Then we have the recurrence
\begin{align*}
  p_1
  &= q + p p_2 \\
  &= q + p p_1^2.
\end{align*}
To see this remark that the increments of the random walk are independent,
thus to arrive at 0 starting at 2
is equivalent to two independent random walks hitting 0 when starting at 1.

Solving this quadratic equation yields solutions $1, \frac qp$.
In the case of $p\geq q$,
we take $\frac qp$
and in the case of $p < q$,
we take 1.
The probability of eventually returning to 0
assuming that $p\geq q$ is thus
\[
  q\cdot 1 + p\cdot \frac qp = 2q
\]
as desired.
To see this,
we condition on the first step being to the left or the right
and apply the appropriate solution to the quadratic equation above.

\clearpage
\section{}
From our work in class,
we know that the expected hitting time for $p=q=\frac12$ is infinity.
But then the expected hitting time for $p<q$ can only be greater
and is thus not finite either.
We focus on the case where $p>q$.

Let us rewrite this as the expected hitting time of 0 starting at position $b$
by flipping the values of $p, q$.
Our strategy to compute this value is to compute the expected hitting time by setting an artificial boundary at $n$
and then letting $n\to \infty$,
similar to our work in class.
This strategy is justified by the monontone convergence theorem.

Let $T_{b, n}$ denote the hitting time of one of two barriers $0, n$
starting at $b\in [0, n]$.
We have
\begin{align*}
  \E[T_{b, n}]
  &= q\E[T_{b-1, n}] + p\E[T_{b+1, n}] + 1 \\
  \E[T_{0, n}] &= 0 \\
  \E[T_{n, n}] &= 0.
\end{align*}

Solving this recurrence under the boundary conditions yield
\begin{align*}
  \E[T_{b, n}]
  &= \frac{b}{q-p} - \frac{n}{q-p}\cdot \frac{\left( \frac qp \right)^b - 1}{\left( \frac qp \right)^n - 1} \\
  &\to \frac{b}{q-p}. &&n\to \infty
\end{align*}

Keeping in mind that we flipped the values of $p, q$,
we conclude that the desired expectation is
\[
  \frac{b}{p-q}
\]
as desired.

\clearpage
\section{}
Write $X_n$ to denote the size of the population at time $n$.
Then
\[
  \P\set{T=n} = \P\set{X_n=0} - \P\set{X_{n-1}=0}
\]
with the base case $\P\set{T=0} = 0$.

Let $\eta_n := \P\set{X_n=0}$.
From our work in class,
\begin{align*}
  \eta_0 &= 0 \\
  \eta_{n+1}
  &= \sum_{k=0}^\infty \eta_n^k f(k) \\
  &= \sum_{k=0}^\infty \eta_n^k\cdot qp^k \\
  &= \frac{q}{1-\eta_n p}.
\end{align*}

Define $r := p/q$.
We argue by induction that
\begin{align*}
  \eta_n =
  \begin{cases}
    \frac{n}{n+1}, &p=q=\frac12 \\
    \frac{r^n-1}{r^{n+1}-1}, &p\neq q
  \end{cases}
\end{align*}

\underline{Case I: $p=q=\frac12$}
We see that the formula is correct for the base case of $n=0$.
Suppose inductively that the formula holds up to some $n\in \N$.
Then
\begin{align*}
  \eta_{n+1}
  &= \frac{q}{1-\eta_n p} \\
  &= \frac{1}{1-\eta_n} \\
  &= \frac{n+1}{2(n+1) - n} \\
  &= \frac{n+1}{n+2}
\end{align*}
as required.

In this case for $n\geq 1$,
\begin{align*}
  \P\set{T=n}
  &= \frac{n}{n+1} - \frac{n-1}{n} \\
  &= \frac{n^2 - (n^2-1)}{n(n+1)} \\
  &= \boxed{\frac{1}{n(n+1)}}. 
\end{align*}

Now,
\begin{align*}
  \E[T]
  &= \sum_{n\geq 1} n\cdot \frac1{n(n+1)} \\
  &= \sum_{n\geq 1} \frac1{n+1}
\end{align*}
is the diverging harmonic series so the expectation is infinite.

\underline{Case II: $p\neq q$}
We see that the formula is correct for the base case of $n=0$.
Suppose inductively that the formula holds up to some $n\in \N$.
Then
\begin{align*}
  \eta_{n+1}
  &= \frac{q}{1-\eta_n p} \\
  &= \frac{q}{1- \frac{r^n-1}{r^{n+1}-1} p} \\
  &= \frac{r^{n+1}-1}{\frac1q (r^{n+1}-1) - (r^n-1)r} \\
  &= \frac{r^{n+1}-1}{r^{n+1} (1/q-1) + (r-1/q)} \\
  &= \frac{r^{n+1}-1}{r^{n+2}-1}
\end{align*}
as desired.

In this case for $n\geq 1$,
\begin{align*}
  \P\set{T=n}
  &= \frac{r^n-1}{r^{n+1}-1} - \frac{r^{n-1}-1}{r^{n}-1} \\
  &= \boxed{\frac{r^{n-1} (r-1)^2}{(r^{n+1}-1)(r^n-1)}}.
\end{align*}

Recall for non-negative discrete random variables
we have the identity
\begin{align*}
  \E[T]
  &= \sum_{n\geq 1} n\cdot \P\set{T=n} \\
  &= \sum_{n\geq 1} \P\set{T\geq n} \\
  &= \sum_{n\geq 1} 1- \P\set{T<n} \\
  &= \sum_{n\geq 1} 1- \P\set{T\leq n-1} \\
  &= \sum_{n\geq 0} 1- \P\set{T\leq n} \\
  &= \sum_{n\geq 0} 1- \P\set{X_n = 0} \\
  &= \sum_{n\geq 0} 1 - \frac{r^{n} - 1}{r^{n+1}-1} \\
  &= \sum_{n\geq 0} \frac{r^{n+1} - 1 - r^{n} + 1}{r^{n+1}-1} \\
  &= \sum_{n\geq 0} \frac{r^n(r - 1)}{r^{n+1}-1}.
\end{align*}

For $r < 1$,
this series converges by the ratio test
\begin{align*}
  \frac{r^{n+1}(r-1)}{r^{n+2}-1}\cdot \frac{r^{n+1}-1}{r^n (r-1)}
  &= \frac{r^{n+2} - r}{r^{n+1}-1} \\
  &\to r \\
  &< 1.
\end{align*}

For $r > 1$,
the series diverges as it is element-wise bounded below by a divergent series
\[
  \frac{r^n (r-1)}{r^{n+1} - 1}
  \geq \frac{r^n (r-1)}{r^{n+1}}
  = \frac{r-1}{r}
  \not\to 0.
\]

\clearpage
\section{}
We use the probability generating function
\[
  \Psi_t(x)
  := \E\left[ x^{G_t} \right]
\]
as the main tool to tackle this problem.

First,
let us show that
\[
  \Psi_{t+1}(x) = \Psi_1(\Psi_t(x)).
\]
Indeed,
consider the more general scenario that $Z = \sum_{k=1}^N Y_k$
where $Y_k, N$ are non-negative discrete variables
such that the $Y_k$'s are iid.
We claim that the generating function $\Psi_Z$ of $Z$ satisfies
\[
  \Psi_Z(x) = \Psi_N(\Psi_Y(x)).
\]
First recall that the generating function of independent random variables is the product
\[
  \Psi_{\sum_{k=1}^m Y_k}(x)
  = \E\left[ x^{\sum_{k=1}^m Y_k} \right]
  = \prod_{k=1}^n \E\left[ x^{Y_k} \right].
\]
Observe here that independence is crucial so that we can factorize the expected values.
We can now compute
\begin{align*}
  \Psi_Y(x)
  &= \sum_{k\geq 0} x^k \P\set{Y = k} \\
  &= \sum_{k\geq 0} x^k \sum_{m\geq 0} \P\set{Y = k\mid N=m}\P\set{N=m} \\
  &= \sum_{m\geq 0} \P\set{N=m}\cdot \sum_{k\geq 0} x^k \P\set{Y = k\mid N=m} \\
  &= \sum_{m\geq 0} \P\set{N=m}\cdot \Psi_{\sum_{k=1}^m Y_k}(x) \\
  &= \sum_{m\geq 0} \P\set{N=m}\cdot \Psi_Y(x)^k \\
  &= \Psi_N(\Psi_Y(x)).
\end{align*}

Recall that we can write
\[
  X_{t+1} = \sum_{k=1}^{X_t} X_{t+1, k}
\]
where $X_{t+1, k}\sim \Po(2)$ iid.
It follows that
\begin{align*}
  \Psi_{t+1}(x)
  &= \Psi_t(\Psi_1(x)) \\
  &\dots \\
  &= (\Psi_1\circ \dots\circ \Psi_1) (x) \\
  &= \Psi_1(\Psi_t(x)).
\end{align*}

Next,
we remark that
\[
  \P\set{G_t=1} = \Psi_t'(0).
\]
To see this observe that
\begin{align*}
  \Psi_t'(0)
  &= \sum_{k\geq 1} k(0)^{k-1} \P\set{G_t=k} \\
  &= \P\set{G_t=1}.
\end{align*}

By the chain rule,
\begin{align*}
  \Psi_{t+1}'(x)
  &= \frac{d}{dx} \Psi_1(\Psi_t(x)) \\
  &= \Psi_1'(\Psi_t(x))\cdot \Psi_t'(x) \\
  \Psi_{t+1}'(0)
  &= \Psi_1'(\Psi_t(0))\cdot \Psi_t'(0).
\end{align*}
It follows that
\begin{align*}
  \rho_t
  &:= \frac{\P\set{G_{t+1} = 1}}{\P\set{G_t=1}} \\
  &= \Psi_1'(\Psi_t(0)) \\
  &= \Psi_1'(\eta_t) &&\eta_t := \P\set{G_t = 0} \\
  &= \sum_{k\geq 1} k\eta_t^{k-1} \P\set{G_1 = k} \\
  &= 2 \sum_{k\geq 1} \frac{(2\eta_t)^{(k-1)} e^{-2}}{(k-1)!} \\
  &= 2 \exp(2\eta_t-2) \\
  &\to 2\exp(2\eta - 2).
\end{align*}

Finally,
from our work in class,
the ultimate extinction probability satisfies
\[
  \eta = \Psi_1(\eta) = \exp(2\eta-2).
\]
Hence
\[
  \boxed{\lim_{t\to \infty} \rho_t = 2\eta}.
\]

\end{document}
