\documentclass[10pt]{article} 

% \usepackage{fullpage}
\usepackage{bookmark}
\usepackage[a4paper,landscape,margin=0.25in]{geometry}
\usepackage{multicol}

\usepackage[dvipsnames]{xcolor}
\usepackage{amsmath}
\usepackage{amssymb}
\usepackage{mathtools}
\usepackage{mathrsfs}
\usepackage{physics}
% \usepackage{unicode-math}
\usepackage{dsfont}

\usepackage[shortlabels]{enumitem}
\usepackage[noabbrev, nameinlink]{cleveref}
\usepackage[most]{tcolorbox}
\usepackage{empheq}
\usepackage[amsmath,standard,thmmarks]{ntheorem} 
\usepackage{bm}
\usepackage{tabularray}
\usepackage{pdfpages}
\usepackage{float}

\setlist[enumerate]{topsep=1pt,itemsep=0pt,partopsep=1ex,parsep=1ex}

% floor, ceiling, set
\DeclarePairedDelimiter{\ceil}{\lceil}{\rceil}
\DeclarePairedDelimiter{\floor}{\lfloor}{\rfloor}
\DeclarePairedDelimiter{\set}{\lbrace}{\rbrace}
\DeclarePairedDelimiter{\iprod}{\langle}{\rangle}
\DeclarePairedDelimiter{\card}{\lvert}{\rvert}
\let\abs\relax
\DeclarePairedDelimiter{\abs}{\lvert}{\rvert}
\let\norm\relax
\DeclarePairedDelimiter{\norm}{\lVert}{\rVert}

\DeclareMathOperator{\Int}{int}
\DeclareMathOperator{\bdy}{bdy}
\DeclareMathOperator{\Lim}{Lim}
\DeclareMathOperator{\mean}{mean}
\DeclareMathOperator{\col}{col}
\DeclareMathOperator{\proj}{proj}
\DeclareMathOperator{\dual}{dual}
\DeclareMathOperator{\opt}{opt}
\DeclareMathOperator{\cone}{cone}
\DeclareMathOperator{\conv}{conv}
\DeclareMathOperator{\supp}{supp}
\DeclareMathOperator{\poly}{poly}
\DeclareMathOperator{\NB}{NB}
\DeclareMathOperator{\Bin}{Bin}
\DeclareMathOperator{\indeg}{indeg}
\DeclareMathOperator{\outdeg}{outdeg}
\DeclareMathOperator{\Var}{Var}
\DeclareMathOperator{\sgn}{sgn}
\DeclareMathOperator{\Span}{span}
\DeclareMathOperator{\odd}{odd}
\DeclareMathOperator{\OPT}{OPT}
\DeclareMathOperator{\aff}{aff}
\DeclareMathOperator{\ri}{ri}
\DeclareMathOperator{\dom}{dom}
\DeclareMathOperator{\epi}{epi}
\DeclareMathOperator{\Id}{Id}
\DeclareMathOperator{\Fix}{Fix}
\DeclareMathOperator{\Prox}{Prox}
\DeclareMathOperator{\argmin}{argmin}
\DeclareMathOperator{\sign}{sign}
\DeclareMathOperator{\zer}{zer}
\DeclareMathOperator{\fl}{fl}
\DeclareMathOperator{\mach}{mach}
\DeclareMathOperator{\KL}{KL}
\DeclareMathOperator{\diag}{diag}
\DeclareMathOperator{\Softmax}{Softmax}
\DeclareMathOperator{\Po}{Po}
\DeclareMathOperator{\Be}{Be}
\DeclareMathOperator{\Cov}{Cov}
\DeclareMathOperator{\Exp}{Exp}

\newcommand{\code}[1]{\lstinline{#1}}

\newcommand{\ones}{\mathds{1}}

\newcommand{\up}{\uparrow}
\newcommand{\down}{\downarrow}
\newcommand{\tends}[1]{\xrightarrow{#1}}
\newcommand{\eq}[1]{\stackrel{#1}{=}}
\newcommand{\Geq}[1]{\stackrel{#1}{\geq}}
\newcommand{\Leq}[1]{\stackrel{#1}{\leq}}

% commonly used sets
\newcommand{\m}{\mathds{m}}
\newcommand{\R}{\mathbb{R}}
\newcommand{\E}{\mathbb{E}}
\newcommand{\Z}{\mathbb{Z}}
\newcommand{\N}{\mathbb{N}}
\newcommand{\Q}{\mathbb{Q}}
\newcommand{\C}{\mathbb{C}}
\renewcommand{\P}{\mathbb{P}}

\newcommand{\B}{\mathcal{B}}
\newcommand{\F}{\mathcal{F}}
\newcommand{\G}{\mathcal{G}}
\newcommand{\U}{\mathcal{U}}
\newcommand{\I}{\mathcal{I}}
\newcommand{\J}{\mathcal{J}}
\renewcommand{\S}{\mathcal{S}}

\newcommand{\W}{\mathbf{W}}
\newcommand{\w}{\mathbf{w}}
\renewcommand{\c}{\mathbf{c}}
\renewcommand{\d}{\mathbf{d}}
\newcommand{\X}{\mathbf{X}}
\newcommand{\x}{\mathbf{x}}
\newcommand{\Y}{\mathbf{Y}}
\newcommand{\y}{\mathbf{y}}
\newcommand{\z}{\mathbf{z}}
\newcommand{\f}{\mathbf{f}}

\newcommand{\h}{\vec{h}}
\newcommand{\p}{\vec{p}}
\renewcommand{\a}{\vec{a}}
\renewcommand{\b}{\vec{b}}
\renewcommand{\t}{\vec{t}}
\renewcommand{\u}{\vec{u}}
\renewcommand{\v}[1]{\vec{#1}}

\newcommand{\sset}{\subseteq}
\newcommand{\mcal}{\mathcal}
\newcommand{\mscr}{\mathscr}
\newcommand{\mbf}{\mathbf}
\newcommand{\mat}[1]{\begin{bmatrix} #1 \end{bmatrix}}
\newcommand{\eff}{\text{eff}}

\newcommand{\NP}{\ensuremath{\mathcal{NP}}}

\newtcbtheorem[
  number within=section,
  crefname={lemma}{Lemma}
]
{lem}
{Lemma}%
{
  theorem style=break,
  sharp corners=all,
  colframe=Red,
  colback={White!95!Red},
  coltitle=black,
  fonttitle=\bfseries,
  beforeafter skip=12pt
}{lem}

\newtcbtheorem[no counter]{pf}{Proof}%
{
  breakable,
  blanker,
  left=5.5mm,
  borderline west={2pt}{0pt}{NavyBlue!80!white},
  after upper=\null\nobreak\hfill\ensuremath{\square},
  colback=white,
  colframe=white,
  coltitle=black,
  fonttitle=\it,
  parbox=false,
  after skip=12pt
}{pf}

\definecolor{dkgreen}{rgb}{0,0.6,0}
\definecolor{gray}{rgb}{0.5,0.5,0.5}
\definecolor{mauve}{rgb}{0.58,0,0.82}
\lstset{
    frame=tb,
    language=Python,
    aboveskip=3mm,
    belowskip=3mm,
    showstringspaces=false,
    columns=flexible,
    basicstyle={\small\ttfamily},
    numbers=none,
    numberstyle=\tiny\color{gray},
    keywordstyle=\color{blue},
    commentstyle=\color{dkgreen},
    stringstyle=\color{mauve},
    breaklines=true,
    breakatwhitespace=true,
    tabsize=3
}

\setlength\parindent{0pt}
\setlength{\parskip}{2pt}

\iffalse
\setcounter{secnumdepth}{2}
\renewcommand\thesection{Problem \arabic{section}.}
\renewcommand\thesubsection{(\alph{subsection})}
\fi


\begin{document}

\iffalse
\begin{center}
    {\Large\textbf{Yale University}}\\
    \vspace{3mm}
    {\Large\textbf{S\&DS 551, Spring 2023}}\\
    \vspace{2mm}
    {\Large\textbf{Homework 3}}\\
    \vspace{3mm}
    \textbf{Chang Feng (Felix) Zhou cz397}
\end{center}
\fi

\begin{multicols}{3}
\section*{S\&DS 551 Midterm Cheat Sheet}
Chang Feng (Felix) Zhou
cz397

\subsection*{Random Walks}
- $S_n :+ S_0 + \sum_{j=1}^n X_j$

- $T$ stopping time boundary $a < b$ \\
- $\P\set{T=\infty\mid S_0=i} = 0$ \\
- $\P\set{S_T = b\mid S_0=i} = \frac{i-a}{b-a}$ \\
- $\E\left[ T\mid S_0=i \right] = (i-a)(b-i)$

- $T_a := \min\set{n: S_n=a}$ \\
- For $0=a<b$,
$\set{S_T=0}\sset \set{T_0<\infty}$ \\
- $\P\set{T_0<\infty\mid S_0=i}\geq \P\set{S_T=0\mid S_0=i} = 1-\frac{i}b\to 1$ \\
- $\E\left[ T_0\mid S_0=i \right] = \lim_{b\to \infty} \E\left[ T\mid S_0=i \right] \to \infty\cdot \delta_{i, 0}$


\subsection*{Branching Processes}
- $X_0 = 1, X_{n+1} = \sum_{i=1}^{X_n} X_{n, i}$
where $X_{n, i}: \Omega\to \N$ is iid

- $\E[X_{n+1}] = \mu, \E[X_n] = \mu^{n+1}, \mu = \E[X_{n, i}]$

- $\eta_n := \P\set{X_n=0}$ \\
- $\eta_{n+1} = \sum_{k\geq 0} p_k \eta_n^k =: g(\eta_n)$ \\
- $\eta = g(\eta)$ fixed point \\
- $g(1) = 1, g'(1) = \mu, g''(x)\geq 0$

\subsection*{Poisson Processes}
- $Z\sim \Po(\lambda)$ \\
- $\P\set{Z=k} = e^{-\lambda} \lambda^k / k!$ \\
- $\E[Z] = \lambda, \Var[Z] = \lambda$

- \emph{Counting Process}: \\
1) $N_0 = 0$,
2) $N_t$ increasing in $t$,
3) Increase at most 1

- \emph{Poisson Process} \\
1) $N_t\sim \Po(\lambda t)$,
2) independent increments,
3) $N_t - N_s\eq{d} N_{t-s}$

- $T_k$: $k$-th arrival time \\
- $N_t$: count at time $t$ \\
- $\set{T_k\leq t} \equiv \set{N_t\geq k}$

- $T_k\sim \Gamma(k, \lambda)$ \\
- $\E[T_k] = k/\lambda, \Var[T_k] = k/\lambda^2$

- $N_t^{(i)}$ iid Poisson process with params $\lambda_i$, \\
$\sum_i N_t^{(i)}$ Poisson process with param $\sum_i \lambda_i$

- $R_t$ Poisson process rate $\lambda$, \\
subsample $k$-th arrival with prob $p\in \Delta^k$, \\
$R_t^{(1)}, \dots, R_t^{(k)}$ ind. Poisson process with params $p_i\lambda$

- $X\sim \Exp(\mu), Y\sim \Exp(\gamma)$ ind.,
$\P\set{X\land Y = X} = \mu/(\mu+\gamma)$

- $N_t$ Poisson process param $\lambda(t)$ if \\
1) independent increments,
2) $N_t - N_s\sim \Po\left( \int_s^t \lambda(u)du \right)$ for $t\geq s$

\subsection*{Markov Chains}
- \emph{Time-homogeneous Markov chain}, \\
$\P\set{X_{n+1}=j\mid \forall 0\leq k\leq n, X_k=i_k} \\
= \P\set{X_{n+1}=j\mid X_n=i_n}$ \\
- \underline{thm:} $P_{ij}(s) = \delta_{ij} + F_{ij}(s) P_{ij}(s)$

- \emph{recurrent state} $i$ if $\P_i\set{\exists n\geq 1, X_n=i} = 1$ \\
- \underline{lem:} If $\sum_n P_{jj}(n) = \infty$, \\
a) $j$ is recurrent,
b) $\sum_n P_{ij}(n) = \infty$ for all $i$ $f_{ij} > 0$

- \emph{transient state} $i$ if $\P_i\set{\exists n\geq 1, X_n=i} < 1$ \\
- \underline{lem:} If $\sum_n P_{jj}(n) < \infty$, \\
a) $j$ is transient,
b) $\sum_n P_{ij}(n) < \infty$ for all $i$ \\
- If $j$ transient,
$\lim_n P_{ij}(n) = 0$ for all $i$

- recurrent $j$ \emph{null-recurrent} if $\E_i \min\set{n\geq 1: X_n=i} = \infty$,
else \emph{positive-recurrent} \\
- $j$ null-recurrent $\iff$ $P_{ii}(n)\to 0$

- \emph{period} state $j$:
$d(j) = \gcd\set{n: P_{jj}(n) > 0}$ \\
$j$ \emph{aperiodic} if $d(j) = 1$

- $i$ communicates with $j$,
$i\to j$ if $\exists m, P_{ij}(m) > 0$

- \underline{thm:} If $i\leftrightarrow j$, \\
1) $d(i) = d(j)$,
2) $i$ transient $\iff$ $j$
3) $i$ null-recurrent $\iff$ $j$

- $C\sset S$ is \emph{closed} if cannot transition out, \\
\emph{irreducible} if $i\leftrightarrow j$ $\forall i, j\in C$ \\
- \underline{lem:} If $\card S < \infty$, \\
a) $\geq 1$ recurrent state,
b) all recurrent states are positive \\
- \underline{thm:} Irreducible MC has stationary distribution
$\iff$ all states positive recurrent,
then $\pi_i = 1/\mu_i$ uniquely (mean recurrent time) \\
- \underline{thm:} Irreducible, aperiodic MC satisfies $P_{ij}(n)\to 1/\mu_j$ \\
- \underline{thm:} Irrreducible, finite, aperiodic MC
has $\lambda_1 = 1$ and $\abs{\lambda_r} < 1$ for all $r > 1$ \\
- $\forall T\sset S,
\sum_{i\in T, j\notin T} \pi_i P_{ij}
= \sum_{i\in T, j\notin T} \pi_j P_{ji}$

- \emph{ergodic:} Irreducible, aperiodic, positive-recurrent MC

- \underline{thm:} RW on $G$ aperiodic $\iff$ $G$ not bipartite \\
- \underline{thm:} RW on conneted non-bipartite graph $\to \pi_v = d(v) / (2\card E)$

- $X_n$ Irreducible, positive-recurrent MC, stationary $\pi$, $X_0\sim \pi$,
\emph{reversed chain} $Y_n := X_{N-m}$ \\
- $Y_n$ is MC with transition $Q_{ij} = \pi_j P_{ji} / \pi_i$ \\
- $X_n$ \emph{reversible} if $\pi_j P_{ji} = \pi_i P_{ij}$ \\
- $X_n$ reversible wrt $\pi$ then $\pi$ stationary distribution

\subsection*{MCMC}
Wish to ``tweak'' MC so that stationary distribution is the desired $\pi$.

- \emph{Metropolis algorithm:} Given MC with symmetric transition $\Psi_{x, y}$, \\
$P_{x, y} =
\begin{cases}
  \Psi_{x, y}\min (1, \pi_y/\pi_x), &y\neq x \\
  1-\sum_{z\neq x} P_{x, z}, &y=x
\end{cases}
$

- \emph{Metropolis-Hastings filter:} Given MC with transition $\Psi_{x, y}$, \\
$P_{x, y} =
\begin{cases}
  \Psi_{x, y}\min (1, \pi_y/\pi_x\cdot \Psi_{y, x}/\Psi_{x, y}), &y\neq x \\
  1-\sum_{z\neq x} P_{x, z}, &y=x
\end{cases}
$

- \emph{Gibbs sampler:} State $S$, $G = (V, E)$,
configurations $\Omega\sset S^V$.
Define $\Omega(x, v) := \set{y\in \Omega: \forall w\neq v, y(w) = x(w)}$. \\
1) Choose $v\in V$ at random, \\
2) $P_{x, y}^v =
\begin{cases}
  \pi(y)/\pi(\Omega(x, v)), &y\in \Omega(x, v) \\
  0, &y\notin \Omega(x, v)
\end{cases}
$

\subsection*{Useful Facts}
- $\sum_{i=1}^n i = n(n+1)/2$

- $\sum_{i=1}^n i^2 = (n+1)(2n+1)/6$

- For $N:\Omega\to \N$, \\
1) $\E[N] = \sum_{n\geq 1} \P\set{N\geq n}$ \\
2) $\E[N^2] = \sum_{n\geq 1} \P\set{N\geq n} \left[ n^2 - (n-1)^2 \right]$

- Catalan number
$C_n = \frac1{n+1}\binom{2n}{n} = \frac{(2n)!}{(n+1)! n!} = \prod_{k=2}^n \frac{n+k}k$

- \emph{Convolution} of seq $a_n, b_n$ is
$c_n := \sum_{i=0}^n a_i b_{n-i}$ \\
generating func $G_c(s) = G_a(s)\cdot G_b(s)$

- Generating func of $X:\Omega\to \N$, \\
$G_X(s) := \E[s^X] =\sum_{i\geq 0} \P\set{X=i} s^i$

- Sum of MC not necessarily a MC
\end{multicols}

\end{document}
